%\documentclass[red, hyperref={pdfpagelabels=false}]{beamer}
\documentclass[red]{beamer}
\hypersetup{pdfpagemode=FullScreen}

\mode<presentation>
\usepackage{beamerthemesplit}
\usepackage[T1]{fontenc}
\usepackage{textcomp}
\usepackage{lmodern}
\usepackage{siunitx}
%\usepackage{listings,bera}
%\usepackage{color}

%\usetheme{boxes}
%\usetheme{Darmstadt}
\usetheme{Dresden}
%\usetheme{Frankfurt}
%\usetheme{Ilmenau}
%\usetheme{Madrid}
%\usetheme{Warsaw}

\title{Examining the Errors in Dual-Polarization Attenuation Correction}
\author{Ryan May}
%\institute{ARRC}
\date{27 February 2014}
\titlegraphic{\includegraphics[scale=0.17]{figures/title_ppi.png}}

\begin{document}

\begin{frame}
	\titlepage
\end{frame}

\begin{frame}{Outline}
    \tableofcontents
\end{frame}

\section{Motivation}
\stepcounter{subsection}
\begin{frame}[<+->]
	\frametitle{Attenuating Systems}
	\begin{itemize}
		\item Proliferation of systems at attenuating wavelengths (C and X bands)
		\item Can be cheaper
		\item Better angular resolution
		\item Better sensitivity
	\end{itemize}
\end{frame}

\begin{frame}
	\frametitle{Applications}
	\begin{itemize}[<+->]
		\item Any quantitative use of the data necessitates correction for attenuation
		\item QPE
		\item Hydrometeor Classification
		\item 3dB error -> 64\% change in rain rate
	\end{itemize}
\end{frame}

\begin{frame}
	\frametitle{Attenuation Correction}
	\begin{itemize}[<+->]
		\item Many different algorithms in the literature
		\item Ones examined here:
		\begin{itemize}
			\item Linear $\Phi_{DP}$
			\item ZPHI
			\item Self-Consistent
		\end{itemize}
		\item All of these rely upon empirically determined coefficients
		\item Which necessitates making certain assumptions (temperature,
		DSD, shape, etc.)
	\end{itemize}
\end{frame}

\begin{frame}[<+->]
	\frametitle{Testing Corrections}
	\begin{itemize}	
		\item Testing of these techniques usually rely upon comparison with
		unattenuated (S-band) data
		\item Another method involves examining reduction in QPE bias upon correction
		\item Instead here we use simulated data to provide data with and without
		attenuation
		\item Simulator allows changing assumptions to see impacts on correction
	\end{itemize}
\end{frame}

\section{Simulation}
\stepcounter{subsection}
\begin{frame}
  \frametitle{Simulator}
  \begin{itemize}[<+->]
  	\item Numerical cloud model output provides needed environmental information
  	\only<1>{\begin{itemize}
  		\item<1> Hydrometeor content
  		\item<1> Wind components
  		\item<1> Thermodynamic information
  		\end{itemize}}
  	\item Propagate discretized radar pulse through simulation grid
  	\item Calculate scattering parameters from hydrometeor information and (optionally) temperature
  	\item During propagation, accumulate phase shift and attenuation
  	\item Use scattering together with radar equation to synthesize radar signal
  	\item Includes antenna pattern (with optional sidelobes), range weighting, and noise
  	\item Also calculate some volume integrated quantities (e.g. attenuation)
  \end{itemize}
\end{frame}

\begin{frame}[<+->]
	\frametitle{Signal Synthesis}
	\begin{itemize}
		\item Each pulse element gets assigned values from the model grid using nearest
		neighbor sampling
		\item Start each element with uniformly distributed random phase
		\item Each element functions as "scattering center" that generates
		a phase shift based on the PRT (Muschinski 1999)
		\item Power at each element is exponentially distributed with expected value
		given from radar equation
		\item Correlation between polarizations uses theoretical $\rho_{HV}$ from scattering calculation (Galati 1995)
		\item Each polarization is attenuated and phase-shifted separately
		\item Altogether, this gives complex, random values for each pulse element for 
		each polarization
		\item These are summed and added to white noise for an IQ sample for the pulse
	\end{itemize}
\end{frame}

\begin{frame}
	\frametitle{Scattering}
	\begin{itemize}
		\item Configurable scattering models
		\begin{itemize}
			\item T-Matrix
			\item Rayleigh-Gans
			\item Mie
			\item Rayleigh
			\end{itemize}
		\item Configurable drop shape model
		\begin{itemize}
			\item Brandes et al. (2004) polynomial model
			\item Pruppacher and Beard (1970) linear model
			\end{itemize}
		\item Wavelength- and temperature-dependent complex dielectric constant
		\item Width of canting angle distribution can also be controlled
	\end{itemize}
\end{frame}

\begin{frame}[<+->]
	\frametitle{Microphysics}
	\begin{itemize}
		\item Native model microphysics scheme is two-moment (Ziegler 1985)
		\item Model distribution is a gamma distribution for volume of drops
		\item However this distribution is not well-suited to radar
		\item Use modified gamma distribution on diameter ($\mu=1.8102$)
		\item This uses the model's number concentration $<D^0>$ and liquid water content $<D^3>$
		\item Also conserves model's $D^6$ (or $V^2$)
	\end{itemize}
\end{frame}

\begin{frame}
	\frametitle{Microphysics (cont.)}
	\begin{center}
		\includegraphics[scale=0.6]{figures/distribution-comparison.pdf}
	\end{center}
\end{frame}

\section{Example Output}
\stepcounter{subsection}
\begin{frame}
	\frametitle{Model Field}
	\begin{center}
		\includegraphics[scale=0.5]{figures/commas_wz_3600_qr_wind_vectors.pdf}	
	\end{center}
\end{frame}

\begin{frame}
	\frametitle{Example Configuration}
	\begin{center}
	    \begin{tabular}{ | l | l | }
	        \hline
	        Antenna gain & \SI{45.5}{dB} \\
	        Peak power & \SI{250}{\kilo\watt} \\
	        First range gate & \SI{500}{\meter} \\
	        Noise power & \SI{-113}{dBm} \\
	        Elevation & \SI{0.5}{\degree} \\
	        PRT & \SI{0.667}{\milli\second} \\
	        Rotation Rate & \SI{20}{\degree\per\second} \\
	        Pulses per radial & \num{75} \\
	        Gate length & \SI{125}{\meter} \\
	        Antenna Limits & Main-lobe only \\
			Wavelength &  \SI{5.5}{\centi\meter} \\
			Beamwidth & \SI{1.0}{\degree} \\
			Radial Spacing & \SI{1.0}{\degree} \\
			\hline
	    \end{tabular}
	\end{center}
\end{frame}

\begin{frame}
	\frametitle{Example Images}
	\includegraphics[scale=0.4]{figures/C_Single.png}
\end{frame}

\begin{frame}
	\frametitle{Example Images (cont.)}
	\includegraphics[scale=0.4]{figures/C_Dual.png}
\end{frame}

\begin{frame}
	\frametitle{Example Images (cont.)}
	\begin{center}
		\includegraphics[scale=0.4]{figures/C_Attenuation.png}
	\end{center}
\end{frame}

\begin{frame}
	\frametitle{Example Images (cont.)}
	\includegraphics[scale=0.4]{figures/X_Single.png}
\end{frame}

\begin{frame}
	\frametitle{Example Images (cont.)}
	\includegraphics[scale=0.4]{figures/X_Dual.png}
\end{frame}

\begin{frame}
	\frametitle{Example Images (cont.)}
	\begin{center}
		\includegraphics[scale=0.4]{figures/X_Attenuation.png}
	\end{center}
\end{frame}

\section{Algorithms}
\stepcounter{subsection}
\begin{frame}
	\frametitle{Linear $\Phi_{DP}$}
	\begin{itemize}
		\item Described by Bringi et al. (1990)
		\item Formulate linear relationship between $K_{DP}$ and $A_H$
		\begin{align*}
			A_H &= \beta_H K_{DP} \\
			A_D &= \beta_D K_{DP}
		\end{align*}
		\item Leads to direct conversion between $\Phi_{DP}$  and  both
				$\alpha_H$ and $\alpha_D$
	\end{itemize}
\end{frame}

\begin{frame}
	\frametitle{ZPHI}
	\begin{itemize}
		\item Described by Testud et al. (2000)
		\item Based around Hitschfeld (1954) reflectivity-based method
		\item Uses linear $\Phi_{DP}$ to provide constraint
			\begin{align*}
			I(r, r_0) &= \num{0.46}b\int_r^{r_0}Z_a^b(s)\,ds \\
			\alpha(r_0) &= \frac{Z_a^b(r_0)}{I(r_1,r_0)} \lbrace 10^{\num{0.1}b\gamma\Delta\Phi} - 1\rbrace \\
			\alpha(r) &= \frac{Z_a^b(r)}{I(r_1,r_0) + \lbrace 10^{\num{0.1}b\gamma\Delta\Phi} - 1\rbrace I(r, r_0)}
			  \times \lbrace 10^{\num{0.1}b\gamma\Delta\Phi} - 1\rbrace
			\end{align*}
	\end{itemize}
\end{frame}

\begin{frame}
	\frametitle{Self-Consistent}
	\begin{itemize}
		\item Described by Bringi et al. (2001)
		\item Adds a "self-consistent" restriction to ZPHI
		\item Attempts to automatically find the optimal value of $\gamma$
			 \begin{align*}
			\phi_{DP}^c(r;\gamma) &= 2 \int_{r_0}^r \frac{A_h(s;\gamma)}{\gamma}\,ds \\
			\gamma_{min} &\leq \gamma \leq \gamma_{max} \\
			Error &= \sum_{j=1}^N \left| \Phi_{DP}^{filt}(r_j) - \Phi_{DP}^c(r_j;\gamma) \right|
			\end{align*}
	\end{itemize}
\end{frame}

\begin{frame}[<+->]
	\frametitle{Coefficient Regression}
	\begin{itemize}
		\item Need to determine own set of coefficients for algorithms
		\item Original algorithms all use different assumptions
		\item All relations some kind of power law
		\item Make best-case coefficients using actual data from model and
		the assumptions used in simulation
	\end{itemize}
\end{frame}

\begin{frame}
	\frametitle{Coefficient Regression (cont.)}
	\begin{center}
		\includegraphics[scale=0.35]{figures/basic_power_law.png}
	\end{center}
	\begin{itemize}
		\item Clearly bias in the fit curve
	\end{itemize}
\end{frame}

\begin{frame}
	\frametitle{Coefficient Regression (cont.)}
	\begin{center}
		\includegraphics[scale=0.35]{figures/weighted_power_law.png}
	\end{center}
	\begin{itemize}
		\item Much better when using $A_H^2$ as the weights in regression
	\end{itemize}
\end{frame}

\section{Errors}
\stepcounter{subsection}
%\subsection{Model Errors}
\begin{frame}
	\frametitle{Experimental Configuration}
	\begin{center}
	    \begin{tabular}{ | l | l | }
	        \hline
	        Antenna gain & \SI{45.5}{dB} \\
	        Peak power & \SI{250}{\kilo\watt} \\
	        First range gate & \SI{500}{\meter} \\
	        Noise power & \SI{-113}{dBm} \\
	        Elevation & \SI{0.5}{\degree} \\
	        PRT & \SI{0.667}{\milli\second} \\
	        Rotation Rate & \SI{20}{\degree\per\second} \\
	        Pulses per radial & \num{75} \\
	        Gate length & \SI{100}{\meter} \\
	        Antenna Limits & Main-lobe only \\
			Beamwidth & \SI{0.25}{\degree} \\
			Radial Spacing & \SI{0.25}{\degree} \\
			Scattering Model & T-Matrix \\
			\hline
	    \end{tabular}
	\end{center}
\end{frame}

\begin{frame}
	\frametitle{Control: Matched Assumptions}
	\begin{center}
	    \begin{tabular}{ | l | l | }
	        \hline
	        Temperature & \SI{283}{\kelvin} \\
	        Drop Shape Model & Brandes \\
	        Wavelength & \SI{5.5}{\centi\meter} \\
			\hline
	    \end{tabular}
	\end{center}	
\end{frame}

\begin{frame}
	\begin{center}
		\includegraphics[scale=0.55]{figures/C_Control_Attenuation.png}
	\end{center}
\end{frame}

\begin{frame}
	\begin{center}
		\includegraphics[scale=0.45]{figures/C_Control_Attenuation_Difference.png}
	\end{center}
\end{frame}

\begin{frame}
	\begin{center}
		\includegraphics[scale=0.55]{figures/C_Control_Specific_Attenuation.png}
	\end{center}
\end{frame}

\begin{frame}
	\begin{center}
		\includegraphics[scale=0.45]{figures/C_Control_Specific_Attenuation_scatter.png}
	\end{center}
\end{frame}

\begin{frame}
	\begin{center}
		\includegraphics[scale=0.45]{figures/C_Control_Differential_Attenuation_Difference.png}
	\end{center}
\end{frame}

\begin{frame}
	\begin{center}
		\includegraphics[scale=0.45]{figures/C_Control_Specific_Differential_Attenuation_scatter.png}
	\end{center}
\end{frame}

\begin{frame}
	\frametitle{Wavelength}
	\begin{center}
	    \begin{tabular}{ | l | l | }
	        \hline
	        Temperature & \SI{283}{\kelvin} \\
	        Drop Shape Model & Brandes \\
	        Wavelength & \SI{5.0}{\centi\meter} \\
			\hline
	    \end{tabular}
	\end{center}	
\end{frame}

\begin{frame}
	\begin{center}
		\includegraphics[scale=0.55]{figures/C_Wavelength_Attenuation.png}
	\end{center}
\end{frame}

\begin{frame}
	\begin{center}
		\includegraphics[scale=0.45]{figures/C_Wavelength_Attenuation_Difference.png}
	\end{center}
\end{frame}

\begin{frame}
	\begin{center}
		\includegraphics[scale=0.55]{figures/C_Wavelength_Specific_Attenuation.png}
	\end{center}
\end{frame}

\begin{frame}
	\begin{center}
		\includegraphics[scale=0.45]{figures/C_Wavelength_Specific_Attenuation_scatter.png}
	\end{center}
\end{frame}

\begin{frame}
	\begin{center}
		\includegraphics[scale=0.45]{figures/C_Wavelength_Differential_Attenuation_Difference.png}
	\end{center}
\end{frame}

\begin{frame}
	\begin{center}
		\includegraphics[scale=0.45]{figures/C_Wavelength_Specific_Differential_Attenuation_scatter.png}
	\end{center}
\end{frame}

\begin{frame}
	\frametitle{Temperature}
	\begin{center}
	    \begin{tabular}{ | l | l | }
	        \hline
	        Temperature & Model Field (\textasciitilde\SI{293}{\kelvin}) \\
	        Drop Shape Model & Brandes \\
	        Wavelength & \SI{5.5}{\centi\meter} \\
			\hline
	    \end{tabular}
	\end{center}	
\end{frame}

\begin{frame}
	\begin{center}
		\includegraphics[scale=0.55]{figures/C_Temperature_Attenuation.png}
	\end{center}
\end{frame}

\begin{frame}
	\begin{center}
		\includegraphics[scale=0.45]{figures/C_Temperature_Attenuation_Difference.png}
	\end{center}
\end{frame}

\begin{frame}
	\begin{center}
		\includegraphics[scale=0.55]{figures/C_Temperature_Specific_Attenuation.png}
	\end{center}
\end{frame}

\begin{frame}
	\begin{center}
		\includegraphics[scale=0.45]{figures/C_Temperature_Specific_Attenuation_scatter.png}
	\end{center}
\end{frame}

\begin{frame}
	\begin{center}
		\includegraphics[scale=0.45]{figures/C_Temperature_Differential_Attenuation_Difference.png}
	\end{center}
\end{frame}

\begin{frame}
	\begin{center}
		\includegraphics[scale=0.45]{figures/C_Temperature_Specific_Differential_Attenuation_scatter.png}
	\end{center}
\end{frame}

\begin{frame}
	\frametitle{Drop Shape Model}
	\begin{center}
	    \begin{tabular}{ | l | l | }
	        \hline
	        Temperature & \SI{283}{\kelvin} \\
	        Drop Shape Model & Pruppacher \\
	        Wavelength & \SI{5.5}{\centi\meter} \\
			\hline
	    \end{tabular}
	\end{center}	
\end{frame}

\begin{frame}
	\begin{center}
		\includegraphics[scale=0.55]{figures/C_Shape_Attenuation.png}
	\end{center}
\end{frame}

\begin{frame}
	\begin{center}
		\includegraphics[scale=0.45]{figures/C_Shape_Attenuation_Difference.png}
	\end{center}
\end{frame}

\begin{frame}
	\begin{center}
		\includegraphics[scale=0.55]{figures/C_Shape_Specific_Attenuation.png}
	\end{center}
\end{frame}

\begin{frame}
	\begin{center}
		\includegraphics[scale=0.45]{figures/C_Shape_Specific_Attenuation_scatter.png}
	\end{center}
\end{frame}

\begin{frame}
	\begin{center}
		\includegraphics[scale=0.45]{figures/C_Shape_Differential_Attenuation_Difference.png}
	\end{center}
\end{frame}

\begin{frame}
	\begin{center}
		\includegraphics[scale=0.45]{figures/C_Shape_Specific_Differential_Attenuation_scatter.png}
	\end{center}
\end{frame}

\begin{frame}
	\frametitle{Worst Case}
	\begin{center}
	    \begin{tabular}{ | l | l | }
	        \hline
	        Temperature & Model Field (\textasciitilde\SI{293}{\kelvin}) \\
	        Drop Shape Model & Pruppacher \\
	        Wavelength & \SI{5.0}{\centi\meter} \\
			\hline
	    \end{tabular}
	\end{center}	
\end{frame}

\begin{frame}
	\begin{center}
		\includegraphics[scale=0.55]{figures/C_Combined_Attenuation.png}
	\end{center}
\end{frame}

\begin{frame}
	\begin{center}
		\includegraphics[scale=0.45]{figures/C_Combined_Attenuation_Difference.png}
	\end{center}
\end{frame}

\begin{frame}
	\begin{center}
		\includegraphics[scale=0.55]{figures/C_Combined_Specific_Attenuation.png}
	\end{center}
\end{frame}

\begin{frame}
	\begin{center}
		\includegraphics[scale=0.45]{figures/C_Combined_Specific_Attenuation_scatter.png}
	\end{center}
\end{frame}

\begin{frame}
	\begin{center}
		\includegraphics[scale=0.45]{figures/C_Combined_Differential_Attenuation_Difference.png}
	\end{center}
\end{frame}

\begin{frame}
	\begin{center}
		\includegraphics[scale=0.45]{figures/C_Combined_Specific_Differential_Attenuation_scatter.png}
	\end{center}
\end{frame}

\section{Conclusion}
\stepcounter{subsection}
\begin{frame}[<+->]
	\frametitle{Conclusions}
	\begin{itemize}
		\item The performance of the linear $\Phi_{DP}$ and ZPHI algorithms depend heavily upon having coefficients
		that reflect the true scattering physics
		\item The self-consistent version of ZPHI performs well overall at removing biases due to changing physics; however this comes
		at a cost of some rays that have increased error
		\item Correction for differential attenuation performs worse across all the algorithms, at least relative to the expected $Z_{DR}$ values
		\item For linear $\Phi_{DP}$ and ZPHI, simply using coefficients to the wave-band of interest can cause non-negligible error
		\item Combining multiple invalid assumptions can actually reduce errors since the biases offset each other
	\end{itemize}
\end{frame}

\begin{frame}
	\frametitle{Thanks}
	\begin{itemize}
		\item Dr. Ted Mansell for performing the model simulation
		\item The NumPy, SciPy, and Matplotlib Python projects that makes this possible
		\item All of my officemates over the years
		\item Dr. Mike Biggerstaff, my advisor, for putting up with me all these years
	\end{itemize}
	\begin{center}
		\large{Questions?}
	\end{center}
\end{frame}

\end{document}
