\documentclass[red, hyperref={pdfpagelabels=false}]{beamer}
%\hypersetup{pdfpagemode=FullScreen}

\mode<presentation>
\usepackage{beamerthemesplit}
\usepackage[T1]{fontenc}
\usepackage{textcomp}
\usepackage{lmodern}
\usepackage{listings,bera}
\usepackage{color}

%\usetheme{boxes}
%\usetheme{Darmstadt}
\usetheme{Dresden}
%\usetheme{Frankfurt}
%\usetheme{Ilmenau}
%\usetheme{Madrid}
%\usetheme{Warsaw}

\title[Porting Radar Simulation Software DRAFT (\today)]
{Porting Radar Simulation Software to Python}
\subtitle{A Case Study in the Benefits of Python}
\author{Ryan May}
\institute{Enterprise Electronics Corporation}
\date{27 January 2011}
\titlegraphic{\includegraphics[scale=0.17]{figures/title_ppi.png}}
\logo{\includegraphics[scale=0.6]{figures/eec_logo_black.png}}

%Python is known for having “batteries included”: a feature-filled standard
%library which reduces development effort for many common tasks, such as logging,
%configuration files, and command line parsing. The utilization of this standard
%library allows the addition of features to software while adding little
%additional code, or even reducing the amount of code for existing software.
%Also, by virtue of its dynamic nature and powerful built-in data structures,
%Python is able to provide a drastically simpler interface for reading NetCDF
%datasets compared to the standard interfaces in languages like C or FORTRAN.
%Python's concept of modules additionally facilitates the creation of small,
%reusable software components, which promotes code reuse. These qualities reduce
%the volume of code that must be developed and maintained, which accelerates the
%development cycle. The porting of a software radar simulator from pure C to a
%mixture of C and Python is used as a case study in the benefits moving software
%to Python.

\begin{document}

\definecolor{keywords}{RGB}{255,0,90}
\definecolor{comments}{RGB}{60,179,113}
\lstset{ %
language=Python,                % choose the language of the code
basicstyle=\scriptsize,         % the size of the fonts that are used for the code
%numbers=left,                   % where to put the line-numbers
%numberstyle=\scriptsize,        % the size of the fonts that are used for the line-numbers
%stepnumber=1,                   % the step between two line-numbers. If it's 1 each line
                                % will be numbered
%numbersep=5pt,                  % how far the line-numbers are from the code
backgroundcolor=\color{white},  % choose the background color. You must add \usepackage{color}
showspaces=false,               % show spaces adding particular underscores
showstringspaces=false,         % underline spaces within strings
showtabs=false,                 % show tabs within strings adding particular underscores
frame=single,                   % adds a frame around the code
tabsize=2,                      % sets default tabsize to 2 spaces
captionpos=b,                   % sets the caption-position to bottom
breaklines=true,                % sets automatic line breaking
breakatwhitespace=false,        % sets if automatic breaks should only happen at whitespace
title=\lstname,                 % show the filename of files included with \lstinputlisting;
                                % also try caption instead of title
escapeinside={\%*}{*)},         % if you want to add a comment within your code
morekeywords={*,...}            % if you want to add more keywords to the set
keywordstyle=\color{keywords},
commentstyle=\color{comments}\emph
}

\frame{\titlepage}

\section[Outline]{}
\begin{frame}{Outline}
    \tableofcontents
\end{frame}

\section{Background}
\stepcounter{subsection}
\begin{frame}
  \frametitle{Simulation Software}
  \begin{itemize}
    \item Software simulates radar data from numerical simulation output
    \item Radar data simulated based on configured radar and scanning strategy parameters
    \item Computationally expensive simulation
    \item Desire to utilize as a teaching tool -> EEK! Other USERS!
    \item Before: 5400 LOC C
    \item Now: 2000 LOC Python, 2900 LOC C (+8500 LOC FORTRAN Library)
    \item Time to "python-ize": ~3 months for a single graduate student (me)
  \end{itemize}
\end{frame}

\begin{frame}
  \frametitle{Why I Chose Python}
  \begin{itemize}
    \item Simple Syntax
    \item Powerful built-in data structures
    \item Excellent string manipulation
    \item Strong built-in library
    \item NumPy - array math
    \item SciPy - scientific algorithms
    \item Matplotlib - plotting
  \end{itemize}
\end{frame}

\section[The Standard Library]{Batteries Included...The Standard Library}
\stepcounter{subsection}
\begin{frame}
  \frametitle{Why is the Standard Library Important?}
  \begin{itemize}
    \item<1-> Libraries you can count on with every installation -> minimal dependency headaches
    \item<2-> Starting point for finding solid libraries to handle given tasks
    \item<3-> Some examples
      \begin{itemize}
        \item re - Regular Expressions
        \item bz2, zipfile, gzip - compressed file handling
        \item os - OS-independent ways for working with files
        \item os.path - OS-independent path/file-name manipulation
        \item subprocess - Running other programs
        \item datetime - working with dates and times
        \item \alert<4->{ConfigParse} - configuration file parsing
        \item \alert<4->{optparse} - command-line option parsing
        \item \alert<4->{logging} - controlling output
      \end{itemize}
  \end{itemize}
\end{frame}

\begin{frame}
  \frametitle{Configuration Files}
  \begin{itemize}
    \item The radar simulation software makes use of configuration files to
      control various radar operation parameters
    \item Made use of ConfigParse module from the standard library
    \item Provides flexible configuration files with support for sections and comments
    \item Simple API to open a configuration file and begin reading options
  \end{itemize}
\end{frame}

\begin{frame}
  \frametitle{Configuration Files:Impacts}
  \begin{itemize}
    \item Replaced brittle file format with something that is easy for users to understand
    \item OLD FORMAT HERE
    \item NEW FORMAT HERE
    \item Replaced brittle, unmaintainable code with something much easier to enhance
    \item Shrank code by 600 LOC
  \end{itemize}
  HUGE usability improvment.
\end{frame}

\begin{frame}
  \frametitle{Command Line Parser}
  \begin{itemize}
    \item optparse module included in Python standard library
    \item Handle parsing of commandline into passed in arguments and options
    \item EXAMPLE COMMANDLINE
    \item EXAMPLE COMPLEX COMMANDLINE
    \item Programmatically add individual options
    \item EXAMPLE CODE HERE
    \item Can also set default values for options
    \item Get a help display for free
    \item SHOW HELP SCREEN HERE
  \end{itemize}
\end{frame}

\begin{frame}[fragile]{Help Screen}
\tiny
\begin{verbatim}
Usage: radarsim [options] configfiles

Options:
  --version             show program's version number and exit
  -h, --help            show this help message and exit
  -v, --verbose         Produce more verbose messages. Specify more than once
                        for more messages.
  -q, --quiet           Make output messages more quiet. Specify more than
                        once for less output.
  -d, --detailed        Use detailed logging messages.
  -l FILE, --log-output=FILE
                        Log output messages to FILE. Only error messages will
                        be displayed to the console.
  -L, --log-only        Used to specify that output only goes to logfile. Need
                        to also specify --log-output.
\end{verbatim}
\end{frame}

\begin{frame}
  \frametitle{Command Line Parser:Impacts}
  \begin{itemize}
    \item Increased code by 100 LOC
    \item However, old code was simplistic
    \item Those 100 lines of new code added support for:
    \begin{itemize}
      \item Controlling how many message are written to screen
      \item Controlling detail of messages
      \item Logging messages to file
      \item Drastically improved help message
    \end{itemize}
  \end{itemize}
  Overall impact was a large increase in usability with minimal development effort.
\end{frame}

\begin{frame}
  \frametitle{Logging}
  \begin{itemize}
    \item Printing messages to the console is necessary to keep tabs on the program.
    \item However, the messages that are useful to me as a developer are not the
      same as what other users needs to see.
    \item Logging libraries provide custom printing functions that allow fine-grained
      control of what messages are printed
    \item Python includes the logging module that includes a raft of features:
    \begin{itemize}
      \item Control of message information (time, filename, line number, etc.)
      \item Different message commands for different levels (warning, info, debug, etc.)
      \item Different message handlers (write to file, write to screen, email?)
    \end{itemize}
  \end{itemize}
\end{frame}

\begin{frame}
  \frametitle{Logging:Impacts}
  \begin{itemize}
    \item Added 40 LOC to set up logging, replaced prints with logger.*
    \item Gained the ability to get detailed messages with a simple command line flag
    \item Gained the ability to only turn on some messages when \emph{I} want them
    \item No \#ifdefs or commenting/uncommenting prints
    \item Big improvements in usability for me as a developer and for other users
  \end{itemize}
\end{frame}

\section{NetCDF}
\stepcounter{subsection}
\begin{frame}
  \frametitle{Python NetCDF API}
  \begin{itemize}
    \item Many packages:
    \begin{itemize}
      \item PyNIO - Part of NCAR's PyNGL package
      \item Scientific.IO.NetCDF
      \item pupynere (scipy.io.netcdf)
      \item python-netcdf4
    \end{itemize}
    \item But they all pretty much follow the same API
    \item Which is \emph{much} simpler than reading and writing NetCDF in C
  \end{itemize}
\end{frame}

\begin{frame}[fragile]{Comparison: Reading a variable}
  \lstset{language=Python}
  \begin{lstlisting}
  from pupynere import netcdf_file
  nc = netcdf_file('test.nc', 'w')
  nc.createDimension('radials', 10)

  az_var = nc.createVariable('Azimuth', 'i', ('radials',))
  az_var.units = 'degrees'
  az_var[:] = range(10)

  nc.close()
  \end{lstlisting}
\end{frame}

\begin{frame}[fragile]{Comparison: Reading a variable}
  \lstset{language=C}
  \begin{lstlisting}
    int ncid;
    int radials_dim;
    int Azimuth_id;
    int Azimuth_dims[1];

    nc_create("test.nc", NC_CLOBBER, &ncid);
    nc_def_dim(ncid, "radials", 10, &radials_dim);

    Azimuth_dims[0] = radials_dim;
    nc_def_var(ncid, "Azimuth", NC_INT, 1, Azimuth_dims, &Azimuth_id);
    nc_put_att_text(ncid, Azimuth_id, "units", 7, "degrees");
    nc_enddef(ncid);
    int Azimuth_data[10] = {0, 1, 2, 3, 4, 5, 6, 7, 8, 9} ;
    nc_put_var_int(ncid, Azimuth_id, Azimuth_data);

    nc_close(ncid);
  \end{lstlisting}
\end{frame}

\begin{frame}
  \frametitle{Comparison: Writing an attribute}
  TODO: Need code samples from C and Python for writing a string attribute on the file (without error handling)
\end{frame}

\begin{frame}
  \frametitle{Python NetCDF API: Impacts}
  \begin{itemize}
    \item Rewrote I/O layer -- shrank by 800 LOC
    \item Better output metadata
    \item Adding new fields and attributes much simpler with Python
    \item As a consequence, the output files contain much more useful
      information for reproducing previous results
  \end{itemize}
\end{frame}

\section[Modularity]{Modularity and Wrapping existing code}
\stepcounter{subsection}
\begin{frame}
  \begin{itemize}
    \item<1-> \alert<2->{Ctypes} - wrap existing library objects (dll's on Windows, .so's on UNIX)
    \item<1-> \alert<2->{F2Py} - call fortran functions from Python
    \item<1-> Cython - Generate C-code (and compiled library) from Python-like syntax
  \end{itemize}
\end{frame}

\begin{frame}
  \begin{itemize}
    \item Integration of existing python scattering code
    \item Same code used to generate plots for other analyses
    \item Re-use leads to better testing
  \end{itemize}
\end{frame}

\section{Wrapping up}
\stepcounter{subsection}
\begin{frame}[<+->]
  \frametitle{Concluding remarks}
  \begin{itemize}
    \item Python offers many built-in libraries that can simplify development...
    \item or allow you to implement features you never would have tried
    \item Python can \emph{drastically} simplify working with NetCDF files
    \item Python allows you keep your performance sensitive code while giving you
      flexibility for those parts that don't need speed (usually most of your code)
    \item Python is \emph{not} necessarily enabling any functionality that could not be done in C or FORTRAN...
    \item but it makes things easier, which allows you to do more.
  \end{itemize}
\end{frame}

\begin{frame}
  \frametitle{Thanks and Questions}
  Thanks to:
  \begin{description}[Matplotlib]
    \item[Python]{http://www.python.org}
    \item[NumPy]{http://numpy.scipy.org}
    \item[SciPy]{http://www.scipy.org}
    \item[Matplotlib]{http://matplotlib.sourceforge.net}
    \item[pupynere]{http://pypi.python.org/pypi/pupynere}
  \end{description}
  Questions?
\end{frame}

\end{document}
