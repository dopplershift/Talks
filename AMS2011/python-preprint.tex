\documentclass[twocolumn]{article}

\usepackage[T1]{fontenc}
\usepackage{textcomp}
\usepackage{lmodern}
\usepackage{listings,bera}
\usepackage{authblk}
\usepackage{times}
\usepackage{color}

\setlength{\affilsep}{0.2em}
\title{Porting Radar Simulation Software to Python: A Case Study in the Benefits of Python}
\author{Ryan May}
\affil{Enterprise Electronics Corporation\\Norman, OK}
\date{}

%Python is known for having “batteries included”: a feature-filled standard
%library which reduces development effort for many common tasks, such as logging,
%configuration files, and command line parsing. The utilization of this standard
%library allows the addition of features to software while adding little
%additional code, or even reducing the amount of code for existing software.
%Also, by virtue of its dynamic nature and powerful built-in data structures,
%Python is able to provide a drastically simpler interface for reading NetCDF
%datasets compared to the standard interfaces in languages like C or FORTRAN.
%Python's concept of modules additionally facilitates the creation of small,
%reusable software components, which promotes code reuse. These qualities reduce
%the volume of code that must be developed and maintained, which accelerates the
%development cycle. The porting of a software radar simulator from pure C to a
%mixture of C and Python is used as a case study in the benefits moving software
%to Python.

\begin{document}
\maketitle

\lstset{ %
language=Python,                % choose the language of the code
basicstyle=\ttfamily\scriptsize, % the size of the fonts that are used for the code
%numbers=left,                   % where to put the line-numbers
%numberstyle=\scriptsize,        % the size of the fonts that are used for the line-numbers
%stepnumber=1,                   % the step between two line-numbers. If it's 1 each line
                                % will be numbered
%numbersep=5pt,                  % how far the line-numbers are from the code
backgroundcolor=\color{white},   % choose the background color. You must add \usepackage{color}
showspaces=false,               % show spaces adding particular underscores
showstringspaces=false,         % underline spaces within strings
showtabs=false,                 % show tabs within strings adding particular underscores
frame=single,                   % adds a frame around the code
tabsize=2,                      % sets default tabsize to 2 spaces
captionpos=b,                   % sets the caption-position to bottom
breaklines=true,                % sets automatic line breaking
breakatwhitespace=true,        % sets if automatic breaks should only happen at whitespace
title=\lstname,                 % show the filename of files included with \lstinputlisting;
                                % also try caption instead of title
escapeinside={\%*}{*)},         % if you want to add a comment within your code
morekeywords={*,...}            % if you want to add more keywords to the set
keywordstyle=\textbf,
commentstyle=\emph,
linewidth=220pt
}


\section{Introduction}
Python, as a dynamic programming language, is frequently used as a "glue"
language--the reference being that Python is used to glue together different parts
of existing software. As a dynamic language, development time tends to be very
quick, at a cost of run-time performance. However, for much of the code that is
written, run-time performance is unimportant; therefore a trade off that
gives rapid development can produce large gains. This is often true in a research
setting, where code turnover is frequent as many different solutions are tried.
Part of this utility also comes from Python's feature-filled standard library
\cite{pythonlib}, which helps a developer find needed functionality without
spending time searching for options.

As a case study in using Python, this work will examine the process of porting
an existing radar simulation software package from straight C to a mixture of C
and Python. Various Python modules used in the porting process and their impacts
on the size, and more importantly, functionality, of the code base will be
discussed. The ease of development for adding these features is also discussed.

\section{Simulation Software}
The software under study is a package for simulating weather radar data,
which uses high resolution (~100m grid spacing) numerical simulation output as
an input source \cite{radarsim}. The simulator relies on a set of configuration
data to control the radar hardware characteristics as well as the operating
parameters. The original code was written in C and heavily optimized for
run-time performance. However, a single simulation still took several hours on a
single processor machine.

Two separate changes in the needs of the software were the motivation behind
incorporating the use of Python. The first was a desire to begin using the
emulator in a classroom setting. Classroom use by students required a large
upgrade to the ease of use so as not to be a support burden for the research
staff. To aid in this role, it was recognized that the terminal interactivity
(commandline invocation and output to the user) and the configuration file
format would need to be improved. The second requirement was the need to upgrade
the code to support simulation of dual-polarimetric radars. This required
incorporating code for more sophisticated scattering models as well as the
capability to select a model at run-time--this is needed to examine the
differences that result from different approximations. Generating simulated
dual-polarimetric data also required changes to the input and output (I/O)
sections of the code to read in new parameters as well as write out new
generated fields. A major consideration in switching to Python was the
simplicity with which NetCDF files (the format used by the simulator) can be
manipulated using Python. All of these changes were constrained to what will be
referred to as the "frontend". The heavily optimized computational core (the
"backend") of the simulator did not require significant changes, and thus kept 
mostly unchanged.

The author's recent (at the time) experience with Python in other domains gave
indications that Python was a promising choice for this additional development.
Adding to this was the fact that the author had already created a flexible
Python module for performing a variety of scattering calculations. Pupynere (or
any of the many other Python NetCDF libraries) provided a clear and simple path
for upgrading the I/O section of the code. Python's standard library, which
contains modules for configuration file parsing, command line parsing, and
logging, implied that the usability improvements could be obtained with no need
to search for reliable libraries to serve these purposes. The standard library
also contains the ctypes module, which served as a straightforward (and
included) way to join the new Python front-end to the C-based backend. The
availability of many features within the standard library was the final tipping
point for deciding to port to Python.

\section{Python Standard Library}
In addition to language syntax and features, part of the rapid development times
in Python comes from its feature-filled standard library \cite{pythonlib}. This
leads many Python developers to refer to Python as having "batteries included."
The standard library gives developers a starting point to look for a given set
of functionality: running external processes, working with files and paths,
regular expressions, etc. In addition to the benefit as a starting point, which
should not be underestimated, the standard library reduces complications from
software dependencies. The modules in the standard library can be relied upon to
be present; thus no need to test for their presence and alert the user to
install if necessary.

Some of the standard library modules used in the simulator include:
\begin{itemize}
    \item collections - item collections
    \item itertools - iterating over groups of items
    \item math - mathematical functions
    \item warnings - issuing and suppressing warnings
    \item os.path - path/file-name manipulation
    \item datetime, time, calendar - dates and times
    \item ConfigParser - configuration file parsing
    \item optparse - command-line option parsing
    \item logging - controlling output
    \item ctypes - calling into dll/.so libraries
\end{itemize}

In this work, we examine specifically the use of standard library modules for:
command line parsing (optparse), configuration files (ConfigParser), and message logging (logging).

\subsection{Configuration Files}
One of the most significant upgrades required to improve the program's usability
was to improve the configuration file format. An example from the original
format is given below:

  \lstinputlisting[title=Old Configuration Format]{oldconfig}

While the format is not too difficult to understand at first glance, it is
somewhat deceptive. Each line is made up of a string with a value. While the
string indicates to the user what the value represents, these strings are not
recognized by the program at all--the lines were simply read in a fixed order.
Any accidental shuffling of the lines, or adding new ones at the wrong location,
would result in cryptic error messages. This is a very unfriendly format for new
users, and prone to errors when new files are created. Additionally, it
was impossible to have any comments in the file to help explain values and
units.

When porting to Python, the standard library's ConfigParser module was selected.
\cite{configparser} This module provides simple and flexible parsing of
configuration files. A sample of the new configuration file format is given
below:

\lstinputlisting[title=New Configuration Format]{newconfig}

The new format is much more flexible, allowing for comments as well as having
no fixed order for the different entries. Additionally, the configuration file
can be broken into separate sections, which can be helpful to organize the file
as well as for the programmer to keep the configuration compartmentalized. This
helps users by having a much more "human-readable" file. From a programmer's
standpoint, the use of ConfigParser yields much simpler and maintainable code.
The actual parsing of lines into keys and values is offloaded to the library, so
that one only needs to request certain parameters by name. Also, the library
supports parsing a flexible (i.e. unknown) number of values as a single
parameter, which was not possible with the old format.

The end result of the conversion of the configuration parsing code from plain C
to using Python's ConfigParser module was a reduction of 600 lines of code
(LOC). The loss in code size represents a sure gain in code maintainability
(fewer lines gives fewer opportunities for bugs). The code's extensibility is
improved as well since adding support for new parameters is as simple as
looking up the new parameter name and having a sensible default if it is not
found. Unspecified parameters using a default would have been exceedingly
difficult, if not impossible, using the old C-based code. All of these gains
together represent a large gain in program usability.

\subsection{Message Logging}
A second change needed to improve the simulator's usability was to improve the
messages displayed to the user.  As a large research code base under heavy
development, the original code displayed a great deal of output to the console.
Such messages would be distracting and/or confusing to the user. The messages
provided useful debugging information, however, so simply removing the messages
was undesirable. What was needed was a way to turn on the messages when
requested but have them off by default.

Python provides the logging module in the standard library \cite{logging}, which
serves the purpose of logging messages. The full framework provides a way of
controlling message formatting, messages at different levels (warn, error,
debug, etc.), and sending individual messages to a variety of end points (files,
screen, email, etc.). Much of the complete functionality was considered overkill
for this application. The only features used here were separate message levels
(for controlling program verbosity) and optionally sending messages to a file
instead of the terminal.

The inclusion of logging messages to a file allows the user to easily collect
a log of the debugging messages, which could be submitted as feedback in the
event of a problem. This streamlines the process of trying to fix issues that
inevitably arise as code gets used more widely under configurations that have
not previously been tested.

The change in the code base to utilize logging were a gain of 40 lines of Python
code. Additionally, all print statements in the C code became calls to logging
functions in Python. For such a small gain in the code base size, useful
features for controlling and saving messages were gained, as well as putting
more polish on the user-visible face of the program.

\subsection{Command Line Parsing}
Besides the configuration file, the command line represents the only user
interface with the simulator. While the original command line interface worked
fine for the original feature set, new options were required to support the
logging module. Also, the original help screen represented essentially a text
file embedded into the source. Any changes to the interface required editing
code in separate location to ensure that the documentation accurately
reflected actual program use.

While the Python standard library's optparse module did not motivate the port to
Python, once the port was underway, it was a clear choice for implementing
command line parsing. optparse provides a clear advantage in terms of creating
code that can easily be extended to support new options \cite{optparse}. Each
option is specified individually, giving:
\begin{itemize}
    \item long and short command names
    \item optional default values
    \item a description of the command (for the program's help message)
    \item what the parser should do when the option is encountered, such as
    \begin{itemize}
        \item Store the subsequent string (such as a filename)
        \item Store true/false
        \item Increment a counter
        \item Run a function
    \end{itemize}
\end{itemize}
From the descriptions of each individual option, the library generates a
help screen for the program, which can be displayed by invoking the program
with the '-h' option or any other case the programmer wishes.

A typical invocation of the program from the command line, which looks like
any typical Linux/UNIX program, is:

{\scriptsize\verb$radarsim -vvv -d -l run.log config/example.config$}

The code used to add an option looks like:

\lstset{language=Python}
\lstinputlisting[caption=]{optparse-large.py}

The use of this module increased the size of the code base by 10 lines. (This does
not include the old hard-coded help page). This additional code added options
for the logging framework that were not available in the old code, as well
as provided a generated, UNIX-like help screen. An example of the help screen
is given below:

{\tiny
\begin{verbatim}
Usage: radarsim [options] configfiles

Options:
  --version             show program's version number and exit
  -h, --help            show this help message and exit
  -v, --verbose         Produce more verbose messages. Specify more than once
                        for more messages.
  -q, --quiet           Make output messages more quiet. Specify more than
                        once for less output.
  -d, --detailed        Use detailed logging messages.
  -l FILE, --log-output=FILE
                        Log output messages to FILE. Only error messages will
                        be displayed to the console.
  -L, --log-only        Used to specify that output only goes to logfile. Need
                        to also specify --log-output.
\end{verbatim}}

All of these changes required minimal development effort. It should be noted
that as of Python's 2.7 release, the optparse library has been deprecated in
favor of the argparse module. This module provides a very similar interface
in many respects, but improves handling of required arguments and default
parameter values. \cite{optparse}

\section{NetCDF}
In addition to the utility of the standard library, the syntax and features of Python
permit library developers to create more expressive interfaces than would be possible
in may other languages. Much of this stems from Python's dynamic nature and
key language features. For instance, Python includes mapping character strings to
arbitrary objects (a dictionary) as one of its core data structures; adding
such support to user-created data structures is very straightforward. That these
mappings can contain arbitrary objects and do not need to be homogeneous is due
to Python's dynamic nature.

NetCDF is a specific example of such a library. The available Python interfaces
for creating NetCDF files are vastly more simple than the reference C interface.
Part of the reason for simplicity is the availability of a de facto standard
array type for Python in the form of the NumPy library. NumPy provides a
mathematical array type that handles memory allocation and provides array-based
mathematical operations \cite{numpy}. The other reason for the simplicity is
again Python's dynamic nature. As a self-describing, flexible file format, the
data types for different fields in NetCDF files are not known a priori. Combined
with the need to know types when compiling, this leads to the creation of large
amount of code (functions, data structures, etc.) to handle the possible cases.
A dynamic language lends itself much more readily to n such a flexible-typed file
format, and this shows in the difference in interfaces.

There exist many Python libraries for reading NetCDF files:
\begin{itemize}
  \item ScientificPython (Scientific.IO.NetCDF)
  \item PyNIO
  \item pupynere
  \item scipy.io.netcdf
  \item netcdf4-python
\end{itemize}
The interface for all of these libraries is nearly identical; ScientificPython's
was the first and the others followed the same interface since it proved
simple to use. For this project, Pupynere was selected as it is implemented
purely in Python \cite{pupynere}, making it easy to install as a dependency; 
there is no need to compile code or install the reference C-based NetCDF library.

As an example of the verbosity in reading a NetCDF file in C, see the listing
below, which retrieves an array of values corresponding to the "Temperature"
variable:

  \lstset{language=C}
  \begin{lstlisting}
    int nc_file, var_id, ndims;
    nc_open("Test_data.nc", NC_NOWRITE, &nc_file);

    nc_inq_varid(nc_file, "Temperature", &var_id);
    nc_inq_varndims(nc_file, var_id, &ndims);
    int* dims = malloc(ndims * sizeof(int));
    nc_inq_vardimid(nc_file, var_id, dims);

    size_t dim_len, total_size = 1;
    for(int i=0; i<ndims; ++i)
    {
        nc_inq_dimlen(nc_file, dims[i], &dim_len);
        total_size *= dim_len;
    }
    float* var = malloc(total_size * sizeof(float));
    nc_get_var_float(nc_file, var_id, var);
 
    free(dims);
    free(var);
    nc_close(nc_file);
  \end{lstlisting}
This stands in stark contrast to the equivalent Python code using pupynere:
  \lstset{language=Python}
  \begin{lstlisting}
  from pupynere import netcdf_file
  nc = netcdf_file('Test_data.nc', 'r')

  temp_var = nc.variables('Temperature')

  nc.close()
  \end{lstlisting}
In the C version of the code, after looking up an ID using the variable's name,
a lot of code is spent figuring out how much memory to allocate to hold the
array.  All of this work is hidden behind a single function call in Python.

Another interesting contrast between the two versions is that to read a different
data type (say Temperature stored as integers), the C version would need to be
updated; the Python version would look identical to the version above. In order
to make a C version that would have the same flexibility, many more lines of
code would be need to check types and to allocate a pointer of the appropriate
type.

While part of Python's advantage comes from  the NumPy library, it would
be possible to write such a data structure and interface functions in C (indeed,
much of the numpy library is in C). However, this would be burdensome on the developer
compared with simply downloading and installing a package. Even if such a C-based
library existed, it still would not have the benefit of being integrated with
the C NetCDF API like pupynere (and others) integrate with NumPy.

It should be noted that these examples are rather simple, as errors are not
handled. However, this again works in favor of Python. In C, the errors are
handled by checking return values of each function invocation, and appropriate
action taken if the error code is not 0. In Python, the task is simpler
since exception handling is available; at the end of the appropriate code block, 
one must simply catch the appropriate exception. This block can encompass
many API calls.

The impact of converting the NetCDF I/O code was a drastic reduction in code
volume: 800 lines of code were removed by converting from C to Python. In
addition, extending the I/O code was made much simpler, which encouraged
some improvements to the metadata which was output in the files. These metadata
include the code version, date and time of simulation, and random number generator
seed (for later reproduction and testing of results).

\section{Porting and Wrapping existing code}
Several methods were considered for wrapping the low-level C code which was to
be kept:
\begin{itemize}
    \item Python C-API
    \item Ctypes
    \item F2Py
    \item Cython
\end{itemize}
The Python C-API was discounted as being too low-level and fragile, as well
as time-consuming to develop and get correct. F2Py's C support was deemed
insufficient for the needs of this project, since its main focus was on FORTRAN.
Cython, (a fork of the Pyrex project), is a tool to generate C code (utilizing
Python's C-API) from a Python-like syntax. This has use both in optimizing
Python code as well as interfacing with external libraries. At the time, the
project was deemed too immature to be relied upon; since that time the project
has gone onto wide success and is used by many scientific Python projects as the
method of choice for interfacing Python with compiled code. Ctypes was chosen as
the mature solution, having gained acceptance into Python's standard library as
of version 2.5. \cite{ctypes}

Ctypes provides a way of opening a shared library (a .so file on UNIX or a .dll
on Windows) and obtaining references to the individual functions contained
within. To pass the proper data structures to these functions, Ctypes provides
a set of classes that map to the various C data types (pointers, int, float, etc.).
These low level types can be assembled into full C-style structures. Once a reference
to the desired function is obtained, one sets the proper types for the function's
arguments as well as its return value. The function is called by creating the
proper structures, filling them with information, and passing them to the function.
This approach has the unique property of having the entire interface to the compiled
code being maintained in Python. It should also be noted that numpy arrays
can be passed as pointers to the appropriate data type using Ctypes.
A simplified version of some of the simulator's Ctypes code is given below:
\lstinputlisting[caption=]{ctypes-example.py}

\section{Conclusion}
The use of Python enabled the rapid development of the additional required
functionality; total development time was approximately 1.5 months for a graduate
student working full-time. Through the use of Python's standard library, the required
features were added without a large increase in code size (and hence maintenance
burden).  In fact, porting to Python, in terms of lines of code, actually reduced
the size of the code base. The initial size of the code was 5400 lines of C code.
The final ported version has 2000 lines of Python and 2900 lines of C (not including
the separate library for scattering calculations). This port came with no measurable
performance penalty; while the new Python code is no doubt slower, the sections of
the code responsible for the overwhelming majority of run time remains in C.
This shows the utility of being able to combine the two languages so that one
can benefit from their respective strengths.

Porting this software package to Python has demonstrated that Python possesses
many attributes that make it easier for developing than in traditional, statically
typed languages. Python's standard library offers many built-in facilities that
make accomplishing a task, like parsing a configuration file, a rapid endeavor.
Python's 3rd-party support for NetCDF files is a large improvement over the
support in C or FORTRAN.  The ability to use Python's Ctypes library (or the
3rd party Cython package) to link to existing code allows one to keep
performance-critical sections of the code in a language like C, while permitting
rapid development in Python for the rest. This allows developers to use the best
tools for the job at hand and maximize their development time. For a scientist,
this means spending less time developing code and more time applying
it to research problems.

\begin{thebibliography}{9}
        \bibitem{pythonlib}
                \emph{The Python Standard Library},
                http://docs.python.org/library/index.html
        \bibitem{radarsim}
                May, Ryan M. and Biggerstaff, Michael I. and Xue, Ming, 2007.
                \emph{A Doppler Radar Emulator with an Application to the Detectability of Tornadic Signatures},
                J. Atmos. Ocean. Tech., \textbf{12}, 1973--1996.
        \bibitem{configparser}
                \emph{ConfigParser -- Configuration file parser},
                http://docs.python.org/library/configparser.html
        \bibitem{logging}
                \emph{logging -- Logging facility for Python},
                http://docs.python.org/library/logging.html
        \bibitem{optparse}
                \emph{optparse -- Parser for command line options},
                http://docs.python.org/library/optparse.html
        \bibitem{numpy}
                \emph{NumPy Reference},
                http://docs.scipy.org/doc/numpy/reference/
        \bibitem{pupynere}
                \emph{pupynere 1.0.13},
                http://pypi.python.org/pypi/pupynere/
        \bibitem{ctypes}
                \emph{ctypes -- A foreign function library for Python},
                http://docs.python.org/library/ctypes.html
\end{thebibliography}

\end{document}
